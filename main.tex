\documentclass[12pt]{article}

% 中文支援
\usepackage{xeCJK}
\setCJKmainfont{Microsoft YaHei} % Windows 內建中文字型(微軟雅黑)
% 或使用:\setCJKmainfont{SimSun} % 宋體
% 或使用:\setCJKmainfont{MingLiU} % 細明體

% 常用套件
\usepackage{amsmath}   % 數學式
\usepackage{amssymb}   % 數學符號
\usepackage{graphicx}  % 插入圖片
\usepackage{geometry}  % 頁面設定
\usepackage{siunitx}  % 加入此行!數字對齊套件
\geometry{a4paper, margin=2.5cm}

\title{我的第一份 \LaTeX{} 文件}
\author{Po-Yin Chang}
\date{\today}

\begin{document}

\maketitle

\section{引言}
本文件示範如何建立第一個 \LaTeX{} 檔案。此工具能以結構化方式撰寫學術論文、簡報與技術文件。

\section{數學式範例}
LaTeX 尤其適合處理數學式。以下為常見的方程式格式:

\subsection{行內數學}
行內格式可使用例如 $E = mc^2$。

\subsection{行間數學}
行間格式的示例如下:
\begin{equation}
    \int_{0}^{1} x^2 \, dx = \frac{1}{3}.
\end{equation}

\subsection{表格範例}
LaTeX 提供了強大的表格功能。以下是一個簡單的表格範例:

\begin{table}[h]
    \centering
    \begin{tabular}{|c|c|c|}
        \hline
        \textbf{項目} & \textbf{數量} & \textbf{價格} \\
        \hline
        蘋果 & 10 & 30 \\
        香蕉 & 5 & 15 \\
        橘子 & 8 & 24 \\
        \hline
    \end{tabular}
    \caption{水果價格表}
    \label{tab:fruits}
\end{table}

\subsection{模型比較範例表格}
以下是不同模型的比較結果:
% 需在前言加入:\usepackage{siunitx}
\begin{table}[h]
    \centering
    \sisetup{table-number-alignment = center}
    \begin{tabular}{l *{4}{S[table-format=1.4]}}
        \hline
        {指標} & {SVR} & {RANDOMFOREST} & {MLP} & {CNN} \\
        \hline
        RMSE  & 0.5    & 0.4     & 0.45    & 0.35 \\
        MSE   & 0.25   & 0.16    & 0.2025  & 0.1225 \\
        MAE   & 0.3    & 0.25    & 0.28    & 0.22 \\
        MAPE  & 5      & 4       & 4.5     & 3.5 \\
        SMAPE & 4      & 3.5     & 3.8     & 3 \\
        \hline
    \end{tabular}
    \caption{不同模型的迴歸指標比較(MAPE/SMAPE 數值單位為%;已對齊小數點)}
    \label{tab:model_comparison}
\end{table}

\section{插入圖片}
若需插入圖片,可使用:
\begin{figure}[h]
    \centering
    \includegraphics[width=0.6\textwidth]{example-image}
    \caption{示意圖標題}
    \label{fig:example}
\end{figure}

\section{結論}
透過本範例,使用者已可建立並編譯一份基本的 \LaTeX{} 文件,後續可擴充至表格、引用、文獻管理等進階功能。

\end{document}
